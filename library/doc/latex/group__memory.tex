\hypertarget{group__memory}{
\subsection{Memory Management}
\label{group__memory}\index{Memory Management@{Memory Management}}
}
Memory management for pMath.  


\subsubsection*{Functions}
\begin{CompactItemize}
\item 
PMATH\_\-API void $\ast$ \hyperlink{group__memory_g856c326c830629de5637912fa8bc2bc9}{pmath\_\-mem\_\-alloc} (size\_\-t size)
\begin{CompactList}\small\item\em Allocate some amount of memory. \item\end{CompactList}\item 
PMATH\_\-API void $\ast$ \hyperlink{group__memory_g59dc67a7de0dc3111dfb0424df8d8244}{pmath\_\-mem\_\-realloc} (void $\ast$p, size\_\-t new\_\-size)
\begin{CompactList}\small\item\em Change the size of a memory-chunk. \item\end{CompactList}\item 
PMATH\_\-API void $\ast$ \hyperlink{group__memory_g212c329fa5e691842e46580bd94f6e4c}{pmath\_\-mem\_\-realloc\_\-no\_\-failfree} (void $\ast$p, size\_\-t new\_\-size)
\begin{CompactList}\small\item\em Change the size of a memory-chunk. \item\end{CompactList}\item 
PMATH\_\-API void \hyperlink{group__memory_g936d3001151c35052812e597eb7dce4f}{pmath\_\-mem\_\-free} (void $\ast$p)
\begin{CompactList}\small\item\em Free a memory-chunk. \item\end{CompactList}\item 
PMATH\_\-API void \hyperlink{group__memory_gfabd51edae8c23c11efdc4540d7b6a0d}{pmath\_\-mem\_\-usage} (size\_\-t $\ast$current, size\_\-t $\ast$max)
\begin{CompactList}\small\item\em Get memory usage information. \item\end{CompactList}\end{CompactItemize}


\subsubsection{Detailed Description}
Memory management for pMath. 

These functions may return NULL. In this case, the current evaluation will abort and used cache will be freed to safe memory (the garbage collector is invoked and a pMath exception is thrown so \hyperlink{group__threads_gb75a9c87401fddb42b297ddb0495415f}{pmath\_\-aborting()} yields TRUE). 

\subsubsection{Function Documentation}
\hypertarget{group__memory_g856c326c830629de5637912fa8bc2bc9}{
\index{memory@{memory}!pmath\_\-mem\_\-alloc@{pmath\_\-mem\_\-alloc}}
\index{pmath\_\-mem\_\-alloc@{pmath\_\-mem\_\-alloc}!memory@{memory}}
\paragraph[{pmath\_\-mem\_\-alloc}]{\setlength{\rightskip}{0pt plus 5cm}PMATH\_\-API void$\ast$ pmath\_\-mem\_\-alloc (size\_\-t {\em size})}\hfill}
\label{group__memory_g856c326c830629de5637912fa8bc2bc9}


Allocate some amount of memory. 

\begin{Desc}
\item[Parameters:]
\begin{description}
\item[{\em size}]The number of bytes to be allocated. \end{description}
\end{Desc}
\begin{Desc}
\item[Returns:]A pointer to a block of mamory of at least size bytes or NULL.\end{Desc}
You must free the result with \hyperlink{group__memory_g936d3001151c35052812e597eb7dce4f}{pmath\_\-mem\_\-free()} or indirectly via \hyperlink{group__memory_g59dc67a7de0dc3111dfb0424df8d8244}{pmath\_\-mem\_\-realloc()}. \hypertarget{group__memory_g59dc67a7de0dc3111dfb0424df8d8244}{
\index{memory@{memory}!pmath\_\-mem\_\-realloc@{pmath\_\-mem\_\-realloc}}
\index{pmath\_\-mem\_\-realloc@{pmath\_\-mem\_\-realloc}!memory@{memory}}
\paragraph[{pmath\_\-mem\_\-realloc}]{\setlength{\rightskip}{0pt plus 5cm}PMATH\_\-API void$\ast$ pmath\_\-mem\_\-realloc (void $\ast$ {\em p}, \/  size\_\-t {\em new\_\-size})}\hfill}
\label{group__memory_g59dc67a7de0dc3111dfb0424df8d8244}


Change the size of a memory-chunk. 

\begin{Desc}
\item[Parameters:]
\begin{description}
\item[{\em p}]NULL or a pointer to a block of old\_\-size bytes allocated with \hyperlink{group__memory_g856c326c830629de5637912fa8bc2bc9}{pmath\_\-mem\_\-alloc()} or \hyperlink{group__memory_g59dc67a7de0dc3111dfb0424df8d8244}{pmath\_\-mem\_\-realloc()}. \item[{\em new\_\-size}]The requested new size. \end{description}
\end{Desc}
\begin{Desc}
\item[Returns:]A pointer to a block of at least new\_\-size bytes or NULL.\end{Desc}
If there is not enough memory available or if new\_\-size == 0, NULL is returned. Otherwise, the result points to a block of new\_\-size bytes, whose first min(old\_\-size,new\_\-size) bytes are copied from the old p. The rest is initialized with 0.

The old pointer p will \_\-allways\_\-be freed. even, if the result is NULL because there is not enough memory.

You must free the result with \hyperlink{group__memory_g936d3001151c35052812e597eb7dce4f}{pmath\_\-mem\_\-free()} or indirectly via \hyperlink{group__memory_g59dc67a7de0dc3111dfb0424df8d8244}{pmath\_\-mem\_\-realloc()}. \hypertarget{group__memory_g212c329fa5e691842e46580bd94f6e4c}{
\index{memory@{memory}!pmath\_\-mem\_\-realloc\_\-no\_\-failfree@{pmath\_\-mem\_\-realloc\_\-no\_\-failfree}}
\index{pmath\_\-mem\_\-realloc\_\-no\_\-failfree@{pmath\_\-mem\_\-realloc\_\-no\_\-failfree}!memory@{memory}}
\paragraph[{pmath\_\-mem\_\-realloc\_\-no\_\-failfree}]{\setlength{\rightskip}{0pt plus 5cm}PMATH\_\-API void$\ast$ pmath\_\-mem\_\-realloc\_\-no\_\-failfree (void $\ast$ {\em p}, \/  size\_\-t {\em new\_\-size})}\hfill}
\label{group__memory_g212c329fa5e691842e46580bd94f6e4c}


Change the size of a memory-chunk. 

\begin{Desc}
\item[Parameters:]
\begin{description}
\item[{\em p}]NULL or a pointer to a block of old\_\-size bytes allocated with \hyperlink{group__memory_g856c326c830629de5637912fa8bc2bc9}{pmath\_\-mem\_\-alloc()} or \hyperlink{group__memory_g59dc67a7de0dc3111dfb0424df8d8244}{pmath\_\-mem\_\-realloc()}. \item[{\em new\_\-size}]The requested new size. \end{description}
\end{Desc}
\begin{Desc}
\item[Returns:]A pointer to a block of at least new\_\-size bytes or NULL.\end{Desc}
If there is enough memory, this acts like \hyperlink{group__memory_g59dc67a7de0dc3111dfb0424df8d8244}{pmath\_\-mem\_\-realloc()}. Otherwise, p is {\em not\/} freed and NULL is returned. \hypertarget{group__memory_g936d3001151c35052812e597eb7dce4f}{
\index{memory@{memory}!pmath\_\-mem\_\-free@{pmath\_\-mem\_\-free}}
\index{pmath\_\-mem\_\-free@{pmath\_\-mem\_\-free}!memory@{memory}}
\paragraph[{pmath\_\-mem\_\-free}]{\setlength{\rightskip}{0pt plus 5cm}PMATH\_\-API void pmath\_\-mem\_\-free (void $\ast$ {\em p})}\hfill}
\label{group__memory_g936d3001151c35052812e597eb7dce4f}


Free a memory-chunk. 

\begin{Desc}
\item[Parameters:]
\begin{description}
\item[{\em p}]NULL or a pointer to a block of old\_\-size bytes allocated with \hyperlink{group__memory_g856c326c830629de5637912fa8bc2bc9}{pmath\_\-mem\_\-alloc()} or \hyperlink{group__memory_g59dc67a7de0dc3111dfb0424df8d8244}{pmath\_\-mem\_\-realloc()}. \end{description}
\end{Desc}
\hypertarget{group__memory_gfabd51edae8c23c11efdc4540d7b6a0d}{
\index{memory@{memory}!pmath\_\-mem\_\-usage@{pmath\_\-mem\_\-usage}}
\index{pmath\_\-mem\_\-usage@{pmath\_\-mem\_\-usage}!memory@{memory}}
\paragraph[{pmath\_\-mem\_\-usage}]{\setlength{\rightskip}{0pt plus 5cm}PMATH\_\-API void pmath\_\-mem\_\-usage (size\_\-t $\ast$ {\em current}, \/  size\_\-t $\ast$ {\em max})}\hfill}
\label{group__memory_gfabd51edae8c23c11efdc4540d7b6a0d}


Get memory usage information. 

\begin{Desc}
\item[Parameters:]
\begin{description}
\item[{\em current}]Here goes the number of currently allocated bytes. \item[{\em max}]here goes the maximum number of allocated bytes until now. \end{description}
\end{Desc}
