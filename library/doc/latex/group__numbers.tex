\hypertarget{group__numbers}{
\subsection{Numbers}
\label{group__numbers}\index{Numbers@{Numbers}}
}
Number objects in pMath.  


\subsubsection*{Data Structures}
\begin{CompactItemize}
\item 
class \hyperlink{classpmath__number__t}{pmath\_\-number\_\-t}
\begin{CompactList}\small\item\em The abstract Number class. \item\end{CompactList}\item 
class \hyperlink{classpmath__rational__t}{pmath\_\-rational\_\-t}
\begin{CompactList}\small\item\em The abstract Rational Number class. \item\end{CompactList}\item 
class \hyperlink{classpmath__integer__t}{pmath\_\-integer\_\-t}
\begin{CompactList}\small\item\em The Integer class. \item\end{CompactList}\item 
class \hyperlink{classpmath__quotient__t}{pmath\_\-quotient\_\-t}
\begin{CompactList}\small\item\em The Quotient class. \item\end{CompactList}\item 
class \hyperlink{classpmath__float__t}{pmath\_\-float\_\-t}
\begin{CompactList}\small\item\em The Floating Point Number class. \item\end{CompactList}\end{CompactItemize}
\subsubsection*{Functions}
\begin{CompactItemize}
\item 
PMATH\_\-API double \hyperlink{group__numbers_g801920d9e9a6b4c04fa45e664a82dc10}{pmath\_\-accuracy} (\hyperlink{classpmath__t}{pmath\_\-t} obj)
\begin{CompactList}\small\item\em Get the accuracy (in bits) of an object. \item\end{CompactList}\item 
PMATH\_\-API double \hyperlink{group__numbers_g8c708b70a1cb0904900187f0c149ab7b}{pmath\_\-precision} (\hyperlink{classpmath__t}{pmath\_\-t} obj)
\begin{CompactList}\small\item\em Get the precision (in bits) of an object. \item\end{CompactList}\item 
PMATH\_\-API \hyperlink{classpmath__t}{pmath\_\-t} \hyperlink{group__numbers_g58e3c6cb9f505bb6c4544dd54de719f0}{pmath\_\-set\_\-accuracy} (\hyperlink{classpmath__t}{pmath\_\-t} obj, double acc)
\begin{CompactList}\small\item\em Set an object's accuracy in bits. \item\end{CompactList}\item 
PMATH\_\-API \hyperlink{classpmath__t}{pmath\_\-t} \hyperlink{group__numbers_g6e44e92e81d1b30567b58568e0151e2c}{pmath\_\-set\_\-precision} (\hyperlink{classpmath__t}{pmath\_\-t} obj, double prec)
\begin{CompactList}\small\item\em Set an object's accuracy in bits. \item\end{CompactList}\item 
PMATH\_\-API \hyperlink{classpmath__t}{pmath\_\-t} \hyperlink{group__numbers_g40f6a60efe25a7a13630d2d51911dc27}{pmath\_\-approximate} (\hyperlink{classpmath__t}{pmath\_\-t} obj, double prec, double acc)
\begin{CompactList}\small\item\em Approximate an object. \item\end{CompactList}\item 
PMATH\_\-API \hyperlink{classpmath__integer__t}{pmath\_\-integer\_\-t} \hyperlink{group__numbers_g34c4815ce444a7dd88a19335098c8c5b}{pmath\_\-integer\_\-t::pmath\_\-integer\_\-new\_\-si} (signed long int si)
\begin{CompactList}\small\item\em Create an integer object from a signed long. \item\end{CompactList}\item 
PMATH\_\-API \hyperlink{classpmath__integer__t}{pmath\_\-integer\_\-t} \hyperlink{group__numbers_ga7704c11e6f72b8403db246f6862d4c6}{pmath\_\-integer\_\-t::pmath\_\-integer\_\-new\_\-ui} (unsigned long int ui)
\begin{CompactList}\small\item\em Create an integer object from an unsigned long. \item\end{CompactList}\item 
PMATH\_\-API \hyperlink{classpmath__integer__t}{pmath\_\-integer\_\-t} \hyperlink{group__numbers_g5fecb78f68ce62c05b1970cf1cf17dc6}{pmath\_\-integer\_\-t::pmath\_\-integer\_\-new\_\-size} (size\_\-t size)
\begin{CompactList}\small\item\em Create an integer object from an size\_\-t. \item\end{CompactList}\item 
PMATH\_\-API \hyperlink{classpmath__integer__t}{pmath\_\-integer\_\-t} \hyperlink{group__numbers_g7ffd98a7d3634c79d261b67f76b1d011}{pmath\_\-integer\_\-t::pmath\_\-integer\_\-new\_\-data} (size\_\-t count, int order, int size, int endian, size\_\-t nails, const void $\ast$data)
\begin{CompactList}\small\item\em Create an integer object from a data buffer. \item\end{CompactList}\item 
PMATH\_\-API \hyperlink{classpmath__integer__t}{pmath\_\-integer\_\-t} \hyperlink{group__numbers_gaca3cb69b051e8f361784f2dc3836fcf}{pmath\_\-integer\_\-t::pmath\_\-integer\_\-new\_\-str} (const char $\ast$str, int base)
\begin{CompactList}\small\item\em Create an integer object from a C String. \item\end{CompactList}\item 
PMATH\_\-API \hyperlink{classpmath__rational__t}{pmath\_\-rational\_\-t} \hyperlink{group__numbers_g142d493a889a7d94bafe79025b61d220}{pmath\_\-rational\_\-t::pmath\_\-rational\_\-new} (\hyperlink{classpmath__integer__t}{pmath\_\-integer\_\-t} numerator, \hyperlink{classpmath__integer__t}{pmath\_\-integer\_\-t} denominator)
\begin{CompactList}\small\item\em Create a rational number. \item\end{CompactList}\item 
PMATH\_\-API \hyperlink{classpmath__integer__t}{pmath\_\-integer\_\-t} \hyperlink{group__numbers_gab6f926fcf84e9cf62c0d99699e491bf}{pmath\_\-rational\_\-t::pmath\_\-rational\_\-numerator} (\hyperlink{classpmath__rational__t}{pmath\_\-rational\_\-t} rational)
\begin{CompactList}\small\item\em Get the numerator of a rational number. \item\end{CompactList}\item 
PMATH\_\-API \hyperlink{classpmath__integer__t}{pmath\_\-integer\_\-t} \hyperlink{group__numbers_ga3f1718d2252ebce77bed83df7223f4c}{pmath\_\-rational\_\-t::pmath\_\-rational\_\-denominator} (\hyperlink{classpmath__rational__t}{pmath\_\-rational\_\-t} rational)
\begin{CompactList}\small\item\em Get the denominator of a rational number. \item\end{CompactList}\item 
PMATH\_\-API \hyperlink{classpmath__number__t}{pmath\_\-number\_\-t} \hyperlink{group__numbers_g6c67d61f1a57084cc46e858a307ec461}{pmath\_\-float\_\-t::pmath\_\-float\_\-new\_\-str} (const char $\ast$str, int base, pmath\_\-precision\_\-control\_\-t precision\_\-control, double base\_\-precision\_\-accuracy)
\begin{CompactList}\small\item\em Create a floating point number from a string. \item\end{CompactList}\item 
PMATH\_\-API \hyperlink{classpmath__float__t}{pmath\_\-float\_\-t} \hyperlink{group__numbers_g5c3beb5316feb510ee46f72f11b49117}{pmath\_\-float\_\-t::pmath\_\-float\_\-new\_\-d} (double dbl)
\begin{CompactList}\small\item\em Create a machine-precision floating point number from a double. \item\end{CompactList}\item 
PMATH\_\-API \hyperlink{group__general__types_gc92090cb0b56345d6c379ed2341d4ef4}{pmath\_\-bool\_\-t} \hyperlink{group__numbers_gc220e06754067b9ba4a7c823017ad32e}{pmath\_\-integer\_\-t::pmath\_\-integer\_\-fits\_\-si} (\hyperlink{classpmath__integer__t}{pmath\_\-integer\_\-t} integer)
\begin{CompactList}\small\item\em Find out whether a pMath integer fits into a signed long int. \item\end{CompactList}\item 
PMATH\_\-API \hyperlink{group__general__types_gc92090cb0b56345d6c379ed2341d4ef4}{pmath\_\-bool\_\-t} \hyperlink{group__numbers_gb12e5e34b7918cb6beee23c57cdd0d36}{pmath\_\-integer\_\-t::pmath\_\-integer\_\-fits\_\-ui} (\hyperlink{classpmath__integer__t}{pmath\_\-integer\_\-t} integer)
\begin{CompactList}\small\item\em Find out whether a pMath integer fits into a unsigned long int. \item\end{CompactList}\item 
PMATH\_\-API signed long int \hyperlink{group__numbers_g12219f6f678ed0ddff66d352e0dabbd1}{pmath\_\-integer\_\-t::pmath\_\-integer\_\-get\_\-si} (\hyperlink{classpmath__integer__t}{pmath\_\-integer\_\-t} integer)
\begin{CompactList}\small\item\em Convert a pMath integer to a signed long int. \item\end{CompactList}\item 
PMATH\_\-API unsigned long int \hyperlink{group__numbers_g0aed3b6f38410b7e42ffa52be73c6ea6}{pmath\_\-integer\_\-t::pmath\_\-integer\_\-get\_\-ui} (\hyperlink{classpmath__integer__t}{pmath\_\-integer\_\-t} integer)
\begin{CompactList}\small\item\em Convert a pMath integer to a unsigned long int. \item\end{CompactList}\item 
PMATH\_\-API double \hyperlink{group__numbers_g62617cbeeedaff88caaafec1dc84f329}{pmath\_\-number\_\-t::pmath\_\-number\_\-get\_\-d} (\hyperlink{classpmath__number__t}{pmath\_\-number\_\-t} number)
\begin{CompactList}\small\item\em Convert a pMath number to a double. \item\end{CompactList}\item 
PMATH\_\-API int \hyperlink{group__numbers_g7a736f04d207d140b3b052438cc371d0}{pmath\_\-number\_\-t::pmath\_\-number\_\-sign} (\hyperlink{classpmath__number__t}{pmath\_\-number\_\-t} num)
\begin{CompactList}\small\item\em Get a number's sign. \item\end{CompactList}\item 
PMATH\_\-API \hyperlink{classpmath__number__t}{pmath\_\-number\_\-t} \hyperlink{group__numbers_gfbfb5ee7f7cd966432e1cee6fa5b6bf5}{pmath\_\-number\_\-t::pmath\_\-number\_\-neg} (\hyperlink{classpmath__number__t}{pmath\_\-number\_\-t} num)
\begin{CompactList}\small\item\em Get a number's negative. \item\end{CompactList}\end{CompactItemize}


\subsubsection{Detailed Description}
Number objects in pMath. 

pMath supports arbitrary big integers and rational values, floating point numbers in machine precision or with automatic precision tracking and complex numbers (the latter are represented by ordinary \hyperlink{classpmath__expr__t}{pmath\_\-expr\_\-t}, all other number types have their own internal representation).

Note that in might be more convinient to use \hyperlink{group__helpers_g13a748aa283c5f5408cce037d3ad224d}{pmath\_\-build\_\-value()} than the specialized constructors represented here, because the former supports Infinity and Indeterminate values for C {\tt double}s.

The GNU Multiple Precision Library (\href{http://gmplib.org/}{\tt http://gmplib.org/}) is used for integer and rational arithmetic and the MPFR library (\href{http://www.mpfr.org/}{\tt http://www.mpfr.org/}) for floating point arithmetic. 

\subsubsection{Function Documentation}
\hypertarget{group__numbers_g801920d9e9a6b4c04fa45e664a82dc10}{
\index{numbers@{numbers}!pmath\_\-accuracy@{pmath\_\-accuracy}}
\index{pmath\_\-accuracy@{pmath\_\-accuracy}!numbers@{numbers}}
\paragraph[{pmath\_\-accuracy}]{\setlength{\rightskip}{0pt plus 5cm}PMATH\_\-API double pmath\_\-accuracy ({\bf pmath\_\-t} {\em obj})}\hfill}
\label{group__numbers_g801920d9e9a6b4c04fa45e664a82dc10}


Get the accuracy (in bits) of an object. 

\begin{Desc}
\item[Parameters:]
\begin{description}
\item[{\em obj}]An object. It will be freed. \end{description}
\end{Desc}
\begin{Desc}
\item[Returns:]The number of known bits after the decimal point.\end{Desc}
HUGE\_\-VAL is given for exact quantities. If {\em obj\/} is an expression, the minimum of its items' accuracies is returned.

Note that the builtin function Accuracy() uses base 10, but this function operates on base 2. \hypertarget{group__numbers_g8c708b70a1cb0904900187f0c149ab7b}{
\index{numbers@{numbers}!pmath\_\-precision@{pmath\_\-precision}}
\index{pmath\_\-precision@{pmath\_\-precision}!numbers@{numbers}}
\paragraph[{pmath\_\-precision}]{\setlength{\rightskip}{0pt plus 5cm}PMATH\_\-API double pmath\_\-precision ({\bf pmath\_\-t} {\em obj})}\hfill}
\label{group__numbers_g8c708b70a1cb0904900187f0c149ab7b}


Get the precision (in bits) of an object. 

\begin{Desc}
\item[Parameters:]
\begin{description}
\item[{\em obj}]An object. It will be freed. \end{description}
\end{Desc}
\begin{Desc}
\item[Returns:]The number of known bits.\end{Desc}
HUGE\_\-VAL is given for exact quantities. -HUGE\_\-VAL means \char`\"{}machine precision\char`\"{}. If {\em obj\/} is an expression, the minimum of its items' accuracies is returned.

Note that the builtin function Precision() uses base 10, but this function operates on base 2. \hypertarget{group__numbers_g58e3c6cb9f505bb6c4544dd54de719f0}{
\index{numbers@{numbers}!pmath\_\-set\_\-accuracy@{pmath\_\-set\_\-accuracy}}
\index{pmath\_\-set\_\-accuracy@{pmath\_\-set\_\-accuracy}!numbers@{numbers}}
\paragraph[{pmath\_\-set\_\-accuracy}]{\setlength{\rightskip}{0pt plus 5cm}PMATH\_\-API {\bf pmath\_\-t} pmath\_\-set\_\-accuracy ({\bf pmath\_\-t} {\em obj}, \/  double {\em acc})}\hfill}
\label{group__numbers_g58e3c6cb9f505bb6c4544dd54de719f0}


Set an object's accuracy in bits. 

\begin{Desc}
\item[Parameters:]
\begin{description}
\item[{\em obj}]An object. It will be freed. \item[{\em acc}]The new number of known bits after the decimal point. \end{description}
\end{Desc}
\begin{Desc}
\item[Returns:]The new object.\end{Desc}
Use {\tt acc == -HUGE\_\-VAL} for machine precision.

Note that the builtin function SetAccuracy() uses base 10, but this function operates on base 2. \hypertarget{group__numbers_g6e44e92e81d1b30567b58568e0151e2c}{
\index{numbers@{numbers}!pmath\_\-set\_\-precision@{pmath\_\-set\_\-precision}}
\index{pmath\_\-set\_\-precision@{pmath\_\-set\_\-precision}!numbers@{numbers}}
\paragraph[{pmath\_\-set\_\-precision}]{\setlength{\rightskip}{0pt plus 5cm}PMATH\_\-API {\bf pmath\_\-t} pmath\_\-set\_\-precision ({\bf pmath\_\-t} {\em obj}, \/  double {\em prec})}\hfill}
\label{group__numbers_g6e44e92e81d1b30567b58568e0151e2c}


Set an object's accuracy in bits. 

\begin{Desc}
\item[Parameters:]
\begin{description}
\item[{\em obj}]An object. It will be freed. \item[{\em prec}]The new number of known bits. \end{description}
\end{Desc}
\begin{Desc}
\item[Returns:]The new object.\end{Desc}
Use {\tt prec == -HUGE\_\-VAL} for machine precision.

Note that the builtin function SetPrecision() uses base 10, but this function operates on base 2. \hypertarget{group__numbers_g40f6a60efe25a7a13630d2d51911dc27}{
\index{numbers@{numbers}!pmath\_\-approximate@{pmath\_\-approximate}}
\index{pmath\_\-approximate@{pmath\_\-approximate}!numbers@{numbers}}
\paragraph[{pmath\_\-approximate}]{\setlength{\rightskip}{0pt plus 5cm}PMATH\_\-API {\bf pmath\_\-t} pmath\_\-approximate ({\bf pmath\_\-t} {\em obj}, \/  double {\em prec}, \/  double {\em acc})}\hfill}
\label{group__numbers_g40f6a60efe25a7a13630d2d51911dc27}


Approximate an object. 

\begin{Desc}
\item[Parameters:]
\begin{description}
\item[{\em obj}]An object. It will be freed. \item[{\em prec}]The requested precision in bits. \item[{\em acc}]The requested accurarcy in bits. \end{description}
\end{Desc}
\begin{Desc}
\item[Returns:]The approximated object.\end{Desc}
Use {\tt prec == -HUGE\_\-VAL} or {\tt acc == -HUGE\_\-VAL} for machine precision. Use {\tt acc == HUGE\_\-VAL} if the accuracy is not imporant and use {\tt prec == HUGE\_\-VAL} if the precision is not important. \hypertarget{group__numbers_g34c4815ce444a7dd88a19335098c8c5b}{
\index{numbers@{numbers}!pmath\_\-integer\_\-new\_\-si@{pmath\_\-integer\_\-new\_\-si}}
\index{pmath\_\-integer\_\-new\_\-si@{pmath\_\-integer\_\-new\_\-si}!numbers@{numbers}}
\paragraph[{pmath\_\-integer\_\-new\_\-si}]{\setlength{\rightskip}{0pt plus 5cm}PMATH\_\-API {\bf pmath\_\-integer\_\-t} pmath\_\-integer\_\-new\_\-si (signed long int {\em si})\hspace{0.3cm}{\tt  \mbox{[}inherited\mbox{]}}}\hfill}
\label{group__numbers_g34c4815ce444a7dd88a19335098c8c5b}


Create an integer object from a signed long. 

\begin{Desc}
\item[Parameters:]
\begin{description}
\item[{\em si}]A signed long int. \end{description}
\end{Desc}
\begin{Desc}
\item[Returns:]A pMath integer with the specified value or NULL. \end{Desc}
\hypertarget{group__numbers_ga7704c11e6f72b8403db246f6862d4c6}{
\index{numbers@{numbers}!pmath\_\-integer\_\-new\_\-ui@{pmath\_\-integer\_\-new\_\-ui}}
\index{pmath\_\-integer\_\-new\_\-ui@{pmath\_\-integer\_\-new\_\-ui}!numbers@{numbers}}
\paragraph[{pmath\_\-integer\_\-new\_\-ui}]{\setlength{\rightskip}{0pt plus 5cm}PMATH\_\-API {\bf pmath\_\-integer\_\-t} pmath\_\-integer\_\-new\_\-ui (unsigned long int {\em ui})\hspace{0.3cm}{\tt  \mbox{[}inherited\mbox{]}}}\hfill}
\label{group__numbers_ga7704c11e6f72b8403db246f6862d4c6}


Create an integer object from an unsigned long. 

\begin{Desc}
\item[Parameters:]
\begin{description}
\item[{\em ui}]An unsigned long int. \end{description}
\end{Desc}
\begin{Desc}
\item[Returns:]A pMath integer with the specified value or NULL. \end{Desc}
\hypertarget{group__numbers_g5fecb78f68ce62c05b1970cf1cf17dc6}{
\index{numbers@{numbers}!pmath\_\-integer\_\-new\_\-size@{pmath\_\-integer\_\-new\_\-size}}
\index{pmath\_\-integer\_\-new\_\-size@{pmath\_\-integer\_\-new\_\-size}!numbers@{numbers}}
\paragraph[{pmath\_\-integer\_\-new\_\-size}]{\setlength{\rightskip}{0pt plus 5cm}PMATH\_\-API {\bf pmath\_\-integer\_\-t} pmath\_\-integer\_\-new\_\-size (size\_\-t {\em size})\hspace{0.3cm}{\tt  \mbox{[}inherited\mbox{]}}}\hfill}
\label{group__numbers_g5fecb78f68ce62c05b1970cf1cf17dc6}


Create an integer object from an size\_\-t. 

\begin{Desc}
\item[Parameters:]
\begin{description}
\item[{\em size}]A size\_\-t value. \end{description}
\end{Desc}
\begin{Desc}
\item[Returns:]A pMath integer with the specified value or NULL.\end{Desc}
Note that on Win64, sizeof(long) == 4, but sizeof(size\_\-t) == 8. \hypertarget{group__numbers_g7ffd98a7d3634c79d261b67f76b1d011}{
\index{numbers@{numbers}!pmath\_\-integer\_\-new\_\-data@{pmath\_\-integer\_\-new\_\-data}}
\index{pmath\_\-integer\_\-new\_\-data@{pmath\_\-integer\_\-new\_\-data}!numbers@{numbers}}
\paragraph[{pmath\_\-integer\_\-new\_\-data}]{\setlength{\rightskip}{0pt plus 5cm}PMATH\_\-API {\bf pmath\_\-integer\_\-t} pmath\_\-integer\_\-new\_\-data (size\_\-t {\em count}, \/  int {\em order}, \/  int {\em size}, \/  int {\em endian}, \/  size\_\-t {\em nails}, \/  const void $\ast$ {\em data})\hspace{0.3cm}{\tt  \mbox{[}inherited\mbox{]}}}\hfill}
\label{group__numbers_g7ffd98a7d3634c79d261b67f76b1d011}


Create an integer object from a data buffer. 

\begin{Desc}
\item[Parameters:]
\begin{description}
\item[{\em count}]The number of words to be read. \item[{\em order}]The order of the words: 1 for most significant word first or -1 for least significant first. \item[{\em size}]The size (in bytes) of a word. \item[{\em endian}]The byte order within each word: 1 for most significant byte first, -1 for least significant first, or 0 for the native endianness of the CPU. \item[{\em nails}]The most significant {\em nails\/} bits of each word are skipped. This can be 0 to use the full words. \item[{\em data}]The buffer to read from. \end{description}
\end{Desc}
\begin{Desc}
\item[Returns:]A non-negative integer.\end{Desc}
\begin{Desc}
\item[See also:]GMP's mpz\_\-import() \end{Desc}
\hypertarget{group__numbers_gaca3cb69b051e8f361784f2dc3836fcf}{
\index{numbers@{numbers}!pmath\_\-integer\_\-new\_\-str@{pmath\_\-integer\_\-new\_\-str}}
\index{pmath\_\-integer\_\-new\_\-str@{pmath\_\-integer\_\-new\_\-str}!numbers@{numbers}}
\paragraph[{pmath\_\-integer\_\-new\_\-str}]{\setlength{\rightskip}{0pt plus 5cm}PMATH\_\-API {\bf pmath\_\-integer\_\-t} pmath\_\-integer\_\-new\_\-str (const char $\ast$ {\em str}, \/  int {\em base})\hspace{0.3cm}{\tt  \mbox{[}inherited\mbox{]}}}\hfill}
\label{group__numbers_gaca3cb69b051e8f361784f2dc3836fcf}


Create an integer object from a C String. 

\begin{Desc}
\item[Parameters:]
\begin{description}
\item[{\em str}]A string representing the value in base {\em base\/}. \item[{\em base}]The base. \end{description}
\end{Desc}
\begin{Desc}
\item[Returns:]A pMath integer with the specified value or NULL.\end{Desc}
See GMP's mpz\_\-set\_\-str for mor information about the parameters. \hypertarget{group__numbers_g142d493a889a7d94bafe79025b61d220}{
\index{numbers@{numbers}!pmath\_\-rational\_\-new@{pmath\_\-rational\_\-new}}
\index{pmath\_\-rational\_\-new@{pmath\_\-rational\_\-new}!numbers@{numbers}}
\paragraph[{pmath\_\-rational\_\-new}]{\setlength{\rightskip}{0pt plus 5cm}PMATH\_\-API {\bf pmath\_\-rational\_\-t} pmath\_\-rational\_\-new ({\bf pmath\_\-integer\_\-t} {\em numerator}, \/  {\bf pmath\_\-integer\_\-t} {\em denominator})\hspace{0.3cm}{\tt  \mbox{[}inherited\mbox{]}}}\hfill}
\label{group__numbers_g142d493a889a7d94bafe79025b61d220}


Create a rational number. 

\begin{Desc}
\item[Parameters:]
\begin{description}
\item[{\em numerator}]The quotient's numerator. It will be freed. \item[{\em denominator}]The quotient's denominator. It will be freed. \end{description}
\end{Desc}
\begin{Desc}
\item[Returns:]An integer, if {\em denominator\/} divides {\em numerator\/} or a quotient in canonical form otherwise. If denominator is zero, NULL will be returned. \end{Desc}
\hypertarget{group__numbers_gab6f926fcf84e9cf62c0d99699e491bf}{
\index{numbers@{numbers}!pmath\_\-rational\_\-numerator@{pmath\_\-rational\_\-numerator}}
\index{pmath\_\-rational\_\-numerator@{pmath\_\-rational\_\-numerator}!numbers@{numbers}}
\paragraph[{pmath\_\-rational\_\-numerator}]{\setlength{\rightskip}{0pt plus 5cm}PMATH\_\-API {\bf pmath\_\-integer\_\-t} pmath\_\-rational\_\-numerator ({\bf pmath\_\-rational\_\-t} {\em rational})\hspace{0.3cm}{\tt  \mbox{[}inherited\mbox{]}}}\hfill}
\label{group__numbers_gab6f926fcf84e9cf62c0d99699e491bf}


Get the numerator of a rational number. 

\begin{Desc}
\item[Parameters:]
\begin{description}
\item[{\em rational}]A rational number (integer or quotient). It wont be freed. \end{description}
\end{Desc}
\begin{Desc}
\item[Returns:]A reference to the numerator of {\em rational\/} if it is a quotient or {\em rational\/} itself if it is an integer. You have to destroy the result e.g. with \hyperlink{classpmath__t_54e905402c38940687033b87eb8c6c9f}{pmath\_\-unref()}. \end{Desc}
\hypertarget{group__numbers_ga3f1718d2252ebce77bed83df7223f4c}{
\index{numbers@{numbers}!pmath\_\-rational\_\-denominator@{pmath\_\-rational\_\-denominator}}
\index{pmath\_\-rational\_\-denominator@{pmath\_\-rational\_\-denominator}!numbers@{numbers}}
\paragraph[{pmath\_\-rational\_\-denominator}]{\setlength{\rightskip}{0pt plus 5cm}PMATH\_\-API {\bf pmath\_\-integer\_\-t} pmath\_\-rational\_\-denominator ({\bf pmath\_\-rational\_\-t} {\em rational})\hspace{0.3cm}{\tt  \mbox{[}inherited\mbox{]}}}\hfill}
\label{group__numbers_ga3f1718d2252ebce77bed83df7223f4c}


Get the denominator of a rational number. 

\begin{Desc}
\item[Parameters:]
\begin{description}
\item[{\em rational}]A rational number (integer or quotient). It wont be freed. \end{description}
\end{Desc}
\begin{Desc}
\item[Returns:]A reference to the denominator of {\em rational\/} if it is a quotient or 1 if it is an integer. You have to destroy the result e.g. with \hyperlink{classpmath__t_54e905402c38940687033b87eb8c6c9f}{pmath\_\-unref()}. \end{Desc}
\hypertarget{group__numbers_g6c67d61f1a57084cc46e858a307ec461}{
\index{numbers@{numbers}!pmath\_\-float\_\-new\_\-str@{pmath\_\-float\_\-new\_\-str}}
\index{pmath\_\-float\_\-new\_\-str@{pmath\_\-float\_\-new\_\-str}!numbers@{numbers}}
\paragraph[{pmath\_\-float\_\-new\_\-str}]{\setlength{\rightskip}{0pt plus 5cm}PMATH\_\-API {\bf pmath\_\-number\_\-t} pmath\_\-float\_\-new\_\-str (const char $\ast$ {\em str}, \/  int {\em base}, \/  pmath\_\-precision\_\-control\_\-t {\em precision\_\-control}, \/  double {\em base\_\-precision\_\-accuracy})\hspace{0.3cm}{\tt  \mbox{[}inherited\mbox{]}}}\hfill}
\label{group__numbers_g6c67d61f1a57084cc46e858a307ec461}


Create a floating point number from a string. 

\begin{Desc}
\item[Parameters:]
\begin{description}
\item[{\em str}]A C-string representing the value in a given {\em base\/}. It should have the form \char`\"{}ddd.ddd\char`\"{} or simply \char`\"{}ddd\char`\"{}. An exponent can be appended with \char`\"{}ennn\char`\"{} or if {\em base\/} != 10 \char`\"{}@nnn\char`\"{}. \item[{\em base}]The base between 2 and 36. \item[{\em precision\_\-control}]flag for controling the precision. \item[{\em base\_\-precision\_\-accuracy}]given precinion or accuracy. depending on the value of the above flag. \end{description}
\end{Desc}
\begin{Desc}
\item[Returns:]a new pMath floating point number or NULL on error or the integer 0 (see below when this happens).\end{Desc}
\begin{Desc}
\item[Remarks:]{\em precision\_\-control\/} may have one of the following values: \begin{itemize}
\item {\tt PMATH\_\-PREC\_\-CTRL\_\-AUTO:} \par
 The precision is specified by the number of digits given in str. It may result in a pMath machine float, mulit-precision float or integer. \par
 The value of {\em base\_\-precision\_\-accuracy\/} will be ignored.

\item {\tt PMATH\_\-PREC\_\-CTRL\_\-MACHINE\_\-PREC:} \par
 The result is a pMath machine float. \par
 The value of {\em base\_\-precision\_\-accuracy\/} will be ignored.

\item {\tt PMATH\_\-PREC\_\-CTRL\_\-GIVEN\_\-PREC:} \par
 If the number's value is 0, the {\em integer\/} 0 will be returned. \par
 The precision is given by {\em base\_\-precision\_\-accuracy\/} (interpreted in the given base).

\item {\tt PMATH\_\-PREC\_\-CTRL\_\-GIVEN\_\-ACC:} \par
 {\em base\_\-precision\_\-accuracy\/} specifies the accuracy (the number of known {\em base\/} -digits after the point). The precision is calculated appropriately. \end{itemize}
\end{Desc}
For a multiprecision float {\tt  x != 0 } with absolute error {\tt dx}, {\tt accuracy} and {\tt precision} are:



\begin{Code}\begin{verbatim}accuracy  = -Log(base, dx)
precision = -Log(base, dx / Abs(x))
\end{verbatim}
\end{Code}



So {\tt precision = accuracy + Log(base, Abs(x))}. \hypertarget{group__numbers_g5c3beb5316feb510ee46f72f11b49117}{
\index{numbers@{numbers}!pmath\_\-float\_\-new\_\-d@{pmath\_\-float\_\-new\_\-d}}
\index{pmath\_\-float\_\-new\_\-d@{pmath\_\-float\_\-new\_\-d}!numbers@{numbers}}
\paragraph[{pmath\_\-float\_\-new\_\-d}]{\setlength{\rightskip}{0pt plus 5cm}PMATH\_\-API {\bf pmath\_\-float\_\-t} pmath\_\-float\_\-new\_\-d (double {\em dbl})\hspace{0.3cm}{\tt  \mbox{[}inherited\mbox{]}}}\hfill}
\label{group__numbers_g5c3beb5316feb510ee46f72f11b49117}


Create a machine-precision floating point number from a double. 

\begin{Desc}
\item[Parameters:]
\begin{description}
\item[{\em dbl}]A double value. \end{description}
\end{Desc}
\begin{Desc}
\item[Returns:]A pMath floating point number with the specified value or NULL. \end{Desc}
\hypertarget{group__numbers_gc220e06754067b9ba4a7c823017ad32e}{
\index{numbers@{numbers}!pmath\_\-integer\_\-fits\_\-si@{pmath\_\-integer\_\-fits\_\-si}}
\index{pmath\_\-integer\_\-fits\_\-si@{pmath\_\-integer\_\-fits\_\-si}!numbers@{numbers}}
\paragraph[{pmath\_\-integer\_\-fits\_\-si}]{\setlength{\rightskip}{0pt plus 5cm}PMATH\_\-API {\bf pmath\_\-bool\_\-t} pmath\_\-integer\_\-fits\_\-si ({\bf pmath\_\-integer\_\-t} {\em integer})\hspace{0.3cm}{\tt  \mbox{[}inherited\mbox{]}}}\hfill}
\label{group__numbers_gc220e06754067b9ba4a7c823017ad32e}


Find out whether a pMath integer fits into a signed long int. 

\begin{Desc}
\item[Parameters:]
\begin{description}
\item[{\em integer}]A pMath integer. It wont be freed. \end{description}
\end{Desc}
\begin{Desc}
\item[Returns:]TRUE iff the value is small enough for a signed long int. \end{Desc}
\hypertarget{group__numbers_gb12e5e34b7918cb6beee23c57cdd0d36}{
\index{numbers@{numbers}!pmath\_\-integer\_\-fits\_\-ui@{pmath\_\-integer\_\-fits\_\-ui}}
\index{pmath\_\-integer\_\-fits\_\-ui@{pmath\_\-integer\_\-fits\_\-ui}!numbers@{numbers}}
\paragraph[{pmath\_\-integer\_\-fits\_\-ui}]{\setlength{\rightskip}{0pt plus 5cm}PMATH\_\-API {\bf pmath\_\-bool\_\-t} pmath\_\-integer\_\-fits\_\-ui ({\bf pmath\_\-integer\_\-t} {\em integer})\hspace{0.3cm}{\tt  \mbox{[}inherited\mbox{]}}}\hfill}
\label{group__numbers_gb12e5e34b7918cb6beee23c57cdd0d36}


Find out whether a pMath integer fits into a unsigned long int. 

\begin{Desc}
\item[Parameters:]
\begin{description}
\item[{\em integer}]A pMath integer. It wont be freed. \end{description}
\end{Desc}
\begin{Desc}
\item[Returns:]TRUE iff the value is small enough for a unsigned long int. \end{Desc}
\hypertarget{group__numbers_g12219f6f678ed0ddff66d352e0dabbd1}{
\index{numbers@{numbers}!pmath\_\-integer\_\-get\_\-si@{pmath\_\-integer\_\-get\_\-si}}
\index{pmath\_\-integer\_\-get\_\-si@{pmath\_\-integer\_\-get\_\-si}!numbers@{numbers}}
\paragraph[{pmath\_\-integer\_\-get\_\-si}]{\setlength{\rightskip}{0pt plus 5cm}PMATH\_\-API signed long int pmath\_\-integer\_\-get\_\-si ({\bf pmath\_\-integer\_\-t} {\em integer})\hspace{0.3cm}{\tt  \mbox{[}inherited\mbox{]}}}\hfill}
\label{group__numbers_g12219f6f678ed0ddff66d352e0dabbd1}


Convert a pMath integer to a signed long int. 

\begin{Desc}
\item[Parameters:]
\begin{description}
\item[{\em integer}]A pMath integer. It wont be freed. \end{description}
\end{Desc}
\begin{Desc}
\item[Returns:]The integer's value if it fits.\end{Desc}
\begin{Desc}
\item[See also:]\hyperlink{group__numbers_gc220e06754067b9ba4a7c823017ad32e}{pmath\_\-integer\_\-fits\_\-si} \end{Desc}
\hypertarget{group__numbers_g0aed3b6f38410b7e42ffa52be73c6ea6}{
\index{numbers@{numbers}!pmath\_\-integer\_\-get\_\-ui@{pmath\_\-integer\_\-get\_\-ui}}
\index{pmath\_\-integer\_\-get\_\-ui@{pmath\_\-integer\_\-get\_\-ui}!numbers@{numbers}}
\paragraph[{pmath\_\-integer\_\-get\_\-ui}]{\setlength{\rightskip}{0pt plus 5cm}PMATH\_\-API unsigned long int pmath\_\-integer\_\-get\_\-ui ({\bf pmath\_\-integer\_\-t} {\em integer})\hspace{0.3cm}{\tt  \mbox{[}inherited\mbox{]}}}\hfill}
\label{group__numbers_g0aed3b6f38410b7e42ffa52be73c6ea6}


Convert a pMath integer to a unsigned long int. 

\begin{Desc}
\item[Parameters:]
\begin{description}
\item[{\em integer}]A pMath integer. It wont be freed. \end{description}
\end{Desc}
\begin{Desc}
\item[Returns:]The integer's value if it fits.\end{Desc}
\begin{Desc}
\item[See also:]\hyperlink{group__numbers_gb12e5e34b7918cb6beee23c57cdd0d36}{pmath\_\-integer\_\-fits\_\-ui} \end{Desc}
\hypertarget{group__numbers_g62617cbeeedaff88caaafec1dc84f329}{
\index{numbers@{numbers}!pmath\_\-number\_\-get\_\-d@{pmath\_\-number\_\-get\_\-d}}
\index{pmath\_\-number\_\-get\_\-d@{pmath\_\-number\_\-get\_\-d}!numbers@{numbers}}
\paragraph[{pmath\_\-number\_\-get\_\-d}]{\setlength{\rightskip}{0pt plus 5cm}PMATH\_\-API double pmath\_\-number\_\-get\_\-d ({\bf pmath\_\-number\_\-t} {\em number})\hspace{0.3cm}{\tt  \mbox{[}inherited\mbox{]}}}\hfill}
\label{group__numbers_g62617cbeeedaff88caaafec1dc84f329}


Convert a pMath number to a double. 

\begin{Desc}
\item[Parameters:]
\begin{description}
\item[{\em number}]A pMath number. It wont be freed. \end{description}
\end{Desc}
\begin{Desc}
\item[Returns:]The number's value if it fits. \end{Desc}
\hypertarget{group__numbers_g7a736f04d207d140b3b052438cc371d0}{
\index{numbers@{numbers}!pmath\_\-number\_\-sign@{pmath\_\-number\_\-sign}}
\index{pmath\_\-number\_\-sign@{pmath\_\-number\_\-sign}!numbers@{numbers}}
\paragraph[{pmath\_\-number\_\-sign}]{\setlength{\rightskip}{0pt plus 5cm}PMATH\_\-API int pmath\_\-number\_\-sign ({\bf pmath\_\-number\_\-t} {\em num})\hspace{0.3cm}{\tt  \mbox{[}inherited\mbox{]}}}\hfill}
\label{group__numbers_g7a736f04d207d140b3b052438cc371d0}


Get a number's sign. 

\begin{Desc}
\item[Parameters:]
\begin{description}
\item[{\em num}]A pMath number. It wont be freed. \end{description}
\end{Desc}
\begin{Desc}
\item[Returns:]The number's sign (-1, 0 or 1) \end{Desc}
\hypertarget{group__numbers_gfbfb5ee7f7cd966432e1cee6fa5b6bf5}{
\index{numbers@{numbers}!pmath\_\-number\_\-neg@{pmath\_\-number\_\-neg}}
\index{pmath\_\-number\_\-neg@{pmath\_\-number\_\-neg}!numbers@{numbers}}
\paragraph[{pmath\_\-number\_\-neg}]{\setlength{\rightskip}{0pt plus 5cm}PMATH\_\-API {\bf pmath\_\-number\_\-t} pmath\_\-number\_\-neg ({\bf pmath\_\-number\_\-t} {\em num})\hspace{0.3cm}{\tt  \mbox{[}inherited\mbox{]}}}\hfill}
\label{group__numbers_gfbfb5ee7f7cd966432e1cee6fa5b6bf5}


Get a number's negative. 

\begin{Desc}
\item[Parameters:]
\begin{description}
\item[{\em num}]A pMath number. It will be freed, do not use it afterwards. \end{description}
\end{Desc}
\begin{Desc}
\item[Returns:]-num \end{Desc}
