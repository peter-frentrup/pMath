\hypertarget{group__helpers}{
\subsection{Object Utility Functions}
\label{group__helpers}\index{Object Utility Functions@{Object Utility Functions}}
}
Utility functuions for pMath Objects and Expressions.  


\subsubsection*{Typedefs}
\begin{CompactItemize}
\item 
typedef \hyperlink{group__general__types_gc92090cb0b56345d6c379ed2341d4ef4}{pmath\_\-bool\_\-t}($\ast$ \hyperlink{group__helpers_g11afc6451921f3169224096723483c64}{pmath\_\-stack\_\-walker\_\-t} )(\hyperlink{classpmath__t}{pmath\_\-t} head, void $\ast$closure)
\begin{CompactList}\small\item\em \hyperlink{class_a}{A} stack walker function. \item\end{CompactList}\end{CompactItemize}
\subsubsection*{Functions}
\begin{CompactItemize}
\item 
PMATH\_\-API \hyperlink{group__general__types_gc92090cb0b56345d6c379ed2341d4ef4}{pmath\_\-bool\_\-t} \hyperlink{group__helpers_g66eb580a000ee1d778b43dec124f170d}{pmath\_\-is\_\-expr\_\-of} (\hyperlink{classpmath__t}{pmath\_\-t} obj, \hyperlink{classpmath__symbol__t}{pmath\_\-symbol\_\-t} head)
\begin{CompactList}\small\item\em Check if an object is an expression with a specified head. \item\end{CompactList}\item 
PMATH\_\-API \hyperlink{group__general__types_gc92090cb0b56345d6c379ed2341d4ef4}{pmath\_\-bool\_\-t} \hyperlink{group__helpers_g68d264fe6b3db0931b4d0a8d5c7427f5}{pmath\_\-is\_\-expr\_\-of\_\-len} (\hyperlink{classpmath__t}{pmath\_\-t} obj, \hyperlink{classpmath__symbol__t}{pmath\_\-symbol\_\-t} head, size\_\-t length)
\begin{CompactList}\small\item\em Check if an object is an expression with a specified head and length. \item\end{CompactList}\item 
PMATH\_\-API \hyperlink{classpmath__t}{pmath\_\-t} \hyperlink{group__helpers_g70aa270956b6c8f8eb43431f9775ae88}{pmath\_\-current\_\-head} (void)
\begin{CompactList}\small\item\em Get the currently evaluated function. \item\end{CompactList}\item 
PMATH\_\-API void \hyperlink{group__helpers_g55213cb0c89952b6aacf90f36ad4047b}{pmath\_\-walk\_\-stack} (\hyperlink{group__helpers_g11afc6451921f3169224096723483c64}{pmath\_\-stack\_\-walker\_\-t} walker, void $\ast$closure)
\begin{CompactList}\small\item\em Walk up the current thread's and its parents' stack. \item\end{CompactList}\item 
PMATH\_\-API void \hyperlink{group__helpers_g077f3730ca4275b87d9a35bce6013e45}{pmath\_\-expr\_\-t::pmath\_\-gather\_\-begin} (\hyperlink{classpmath__t}{pmath\_\-t} pattern)
\begin{CompactList}\small\item\em Start gathering emitted objects. \item\end{CompactList}\item 
PMATH\_\-API \hyperlink{classpmath__expr__t}{pmath\_\-expr\_\-t} \hyperlink{group__helpers_ga2f732b35703986263e3a15592b4a46e}{pmath\_\-expr\_\-t::pmath\_\-gather\_\-end} (void)
\begin{CompactList}\small\item\em Finish gathering emitted objects. \item\end{CompactList}\item 
PMATH\_\-API void \hyperlink{group__helpers_ga06135012f4d2a0faf696c0cd1111075}{pmath\_\-expr\_\-t::pmath\_\-emit} (\hyperlink{classpmath__t}{pmath\_\-t} object, \hyperlink{classpmath__t}{pmath\_\-t} tag)
\begin{CompactList}\small\item\em Emit an object to be gathered by the appropriate surounding \hyperlink{group__helpers_g077f3730ca4275b87d9a35bce6013e45}{pmath\_\-gather\_\-begin()} ... \hyperlink{group__helpers_ga2f732b35703986263e3a15592b4a46e}{pmath\_\-gather\_\-end()} function pair. \item\end{CompactList}\item 
PMATH\_\-API \hyperlink{classpmath__t}{pmath\_\-t} \hyperlink{group__helpers_g96660afa0732edf675653eb6ddeab0a4}{pmath\_\-t::pmath\_\-build\_\-value\_\-v} (const char $\ast$format, va\_\-list args)
\begin{CompactList}\small\item\em Generate a List of objects with a format string. \item\end{CompactList}\item 
PMATH\_\-API \hyperlink{classpmath__t}{pmath\_\-t} \hyperlink{group__helpers_g13a748aa283c5f5408cce037d3ad224d}{pmath\_\-t::pmath\_\-build\_\-value} (const char $\ast$format,...)
\begin{CompactList}\small\item\em Generate a List of objects with a format string. \item\end{CompactList}\item 
PMATH\_\-API \hyperlink{classpmath__expr__t}{pmath\_\-expr\_\-t} \hyperlink{group__helpers_g24403dfbd825b17fc4c6da5973922184}{pmath\_\-expr\_\-t::pmath\_\-options\_\-extract} (\hyperlink{classpmath__expr__t}{pmath\_\-expr\_\-t} expr, size\_\-t last\_\-nonoption)
\begin{CompactList}\small\item\em Extract custom option values from an expression. \item\end{CompactList}\item 
PMATH\_\-API \hyperlink{classpmath__t}{pmath\_\-t} \hyperlink{group__helpers_gc244ab0720278b396976728a39f8bde6}{pmath\_\-expr\_\-t::pmath\_\-option\_\-value} (\hyperlink{classpmath__t}{pmath\_\-t} fn, \hyperlink{classpmath__t}{pmath\_\-t} name, \hyperlink{classpmath__t}{pmath\_\-t} extra)
\begin{CompactList}\small\item\em Retrieve a option value of a given function. \item\end{CompactList}\end{CompactItemize}


\subsubsection{Detailed Description}
Utility functuions for pMath Objects and Expressions. 

Here are some utility functions that simplify access to Expressions (or pMath Objects in general), but do not realy fit one of these topics. 

\subsubsection{Typedef Documentation}
\hypertarget{group__helpers_g11afc6451921f3169224096723483c64}{
\index{helpers@{helpers}!pmath\_\-stack\_\-walker\_\-t@{pmath\_\-stack\_\-walker\_\-t}}
\index{pmath\_\-stack\_\-walker\_\-t@{pmath\_\-stack\_\-walker\_\-t}!helpers@{helpers}}
\paragraph[{pmath\_\-stack\_\-walker\_\-t}]{\setlength{\rightskip}{0pt plus 5cm}typedef {\bf pmath\_\-bool\_\-t}($\ast$ {\bf pmath\_\-stack\_\-walker\_\-t})({\bf pmath\_\-t} head, void $\ast$closure)}\hfill}
\label{group__helpers_g11afc6451921f3169224096723483c64}


\hyperlink{class_a}{A} stack walker function. 

The return value specifies, whether the walk on the stack go on. 

\subsubsection{Function Documentation}
\hypertarget{group__helpers_g66eb580a000ee1d778b43dec124f170d}{
\index{helpers@{helpers}!pmath\_\-is\_\-expr\_\-of@{pmath\_\-is\_\-expr\_\-of}}
\index{pmath\_\-is\_\-expr\_\-of@{pmath\_\-is\_\-expr\_\-of}!helpers@{helpers}}
\paragraph[{pmath\_\-is\_\-expr\_\-of}]{\setlength{\rightskip}{0pt plus 5cm}PMATH\_\-API {\bf pmath\_\-bool\_\-t} pmath\_\-is\_\-expr\_\-of ({\bf pmath\_\-t} {\em obj}, \/  {\bf pmath\_\-symbol\_\-t} {\em head})}\hfill}
\label{group__helpers_g66eb580a000ee1d778b43dec124f170d}


Check if an object is an expression with a specified head. 

\begin{Desc}
\item[Parameters:]
\begin{description}
\item[{\em obj}]\hyperlink{class_a}{A} pMath object. It wont be freed. \item[{\em head}]\hyperlink{class_a}{A} pMath symbol. It wont be freed. \end{description}
\end{Desc}
\begin{Desc}
\item[Returns:]TRUE iff obj is an expression with the given {\em head\/} which must be a symbol. \end{Desc}
\hypertarget{group__helpers_g68d264fe6b3db0931b4d0a8d5c7427f5}{
\index{helpers@{helpers}!pmath\_\-is\_\-expr\_\-of\_\-len@{pmath\_\-is\_\-expr\_\-of\_\-len}}
\index{pmath\_\-is\_\-expr\_\-of\_\-len@{pmath\_\-is\_\-expr\_\-of\_\-len}!helpers@{helpers}}
\paragraph[{pmath\_\-is\_\-expr\_\-of\_\-len}]{\setlength{\rightskip}{0pt plus 5cm}PMATH\_\-API {\bf pmath\_\-bool\_\-t} pmath\_\-is\_\-expr\_\-of\_\-len ({\bf pmath\_\-t} {\em obj}, \/  {\bf pmath\_\-symbol\_\-t} {\em head}, \/  size\_\-t {\em length})}\hfill}
\label{group__helpers_g68d264fe6b3db0931b4d0a8d5c7427f5}


Check if an object is an expression with a specified head and length. 

\begin{Desc}
\item[Parameters:]
\begin{description}
\item[{\em obj}]\hyperlink{class_a}{A} pMath object. It wont be freed. \item[{\em head}]\hyperlink{class_a}{A} pMath symbol. It wont be freed. \item[{\em length}]The requested expression length. \end{description}
\end{Desc}
\begin{Desc}
\item[Returns:]TRUE iff obj is an expression with the given {\em length\/} and {\em head\/} which must be a symbol. \end{Desc}
\hypertarget{group__helpers_g70aa270956b6c8f8eb43431f9775ae88}{
\index{helpers@{helpers}!pmath\_\-current\_\-head@{pmath\_\-current\_\-head}}
\index{pmath\_\-current\_\-head@{pmath\_\-current\_\-head}!helpers@{helpers}}
\paragraph[{pmath\_\-current\_\-head}]{\setlength{\rightskip}{0pt plus 5cm}PMATH\_\-API {\bf pmath\_\-t} pmath\_\-current\_\-head (void)}\hfill}
\label{group__helpers_g70aa270956b6c8f8eb43431f9775ae88}


Get the currently evaluated function. 

\begin{Desc}
\item[Returns:]The head of the expression that is currently evaluated (in the calling thread). You have to destroy it. \end{Desc}
\hypertarget{group__helpers_g55213cb0c89952b6aacf90f36ad4047b}{
\index{helpers@{helpers}!pmath\_\-walk\_\-stack@{pmath\_\-walk\_\-stack}}
\index{pmath\_\-walk\_\-stack@{pmath\_\-walk\_\-stack}!helpers@{helpers}}
\paragraph[{pmath\_\-walk\_\-stack}]{\setlength{\rightskip}{0pt plus 5cm}PMATH\_\-API void pmath\_\-walk\_\-stack ({\bf pmath\_\-stack\_\-walker\_\-t} {\em walker}, \/  void $\ast$ {\em closure})}\hfill}
\label{group__helpers_g55213cb0c89952b6aacf90f36ad4047b}


Walk up the current thread's and its parents' stack. 

\begin{Desc}
\item[Parameters:]
\begin{description}
\item[{\em walker}]\hyperlink{class_a}{A} callback function. \item[{\em closure}]\hyperlink{class_a}{A} pointer that will be provided to walker as the second argument. \end{description}
\end{Desc}
\hypertarget{group__helpers_g077f3730ca4275b87d9a35bce6013e45}{
\index{helpers@{helpers}!pmath\_\-gather\_\-begin@{pmath\_\-gather\_\-begin}}
\index{pmath\_\-gather\_\-begin@{pmath\_\-gather\_\-begin}!helpers@{helpers}}
\paragraph[{pmath\_\-gather\_\-begin}]{\setlength{\rightskip}{0pt plus 5cm}PMATH\_\-API void pmath\_\-gather\_\-begin ({\bf pmath\_\-t} {\em pattern})\hspace{0.3cm}{\tt  \mbox{[}related, inherited\mbox{]}}}\hfill}
\label{group__helpers_g077f3730ca4275b87d9a35bce6013e45}


Start gathering emitted objects. 

\begin{Desc}
\item[Parameters:]
\begin{description}
\item[{\em pattern}]\hyperlink{class_a}{A} pattern that is used to determine which emitted objects should be gathered (testing the emit-tag, not the object itself). It will be freed.\end{description}
\end{Desc}
Use \hyperlink{group__helpers_ga2f732b35703986263e3a15592b4a46e}{pmath\_\-gather\_\-end()} to finish gathering. Calls to \hyperlink{group__helpers_g077f3730ca4275b87d9a35bce6013e45}{pmath\_\-gather\_\-begin()} ... \hyperlink{group__helpers_ga2f732b35703986263e3a15592b4a46e}{pmath\_\-gather\_\-end()} can be nested.

The emit-and-gather mechanism is useful when you want to create a list but do not know its final length in advance.

\begin{Desc}
\item[See also:]\hyperlink{group__helpers_ga06135012f4d2a0faf696c0cd1111075}{pmath\_\-emit} \end{Desc}
\hypertarget{group__helpers_ga2f732b35703986263e3a15592b4a46e}{
\index{helpers@{helpers}!pmath\_\-gather\_\-end@{pmath\_\-gather\_\-end}}
\index{pmath\_\-gather\_\-end@{pmath\_\-gather\_\-end}!helpers@{helpers}}
\paragraph[{pmath\_\-gather\_\-end}]{\setlength{\rightskip}{0pt plus 5cm}PMATH\_\-API {\bf pmath\_\-expr\_\-t} pmath\_\-gather\_\-end (void)\hspace{0.3cm}{\tt  \mbox{[}related, inherited\mbox{]}}}\hfill}
\label{group__helpers_ga2f732b35703986263e3a15592b4a46e}


Finish gathering emitted objects. 

\begin{Desc}
\item[Returns:]\hyperlink{class_a}{A} list of all emitted objects since the last \hyperlink{group__helpers_g077f3730ca4275b87d9a35bce6013e45}{pmath\_\-gather\_\-begin()} whose emit-tag matched the {\tt pattern} parameter given to that \hyperlink{group__helpers_g077f3730ca4275b87d9a35bce6013e45}{pmath\_\-gather\_\-begin()}. You must free it.\end{Desc}
\begin{Desc}
\item[See also:]\hyperlink{group__helpers_ga06135012f4d2a0faf696c0cd1111075}{pmath\_\-emit} \end{Desc}
\hypertarget{group__helpers_ga06135012f4d2a0faf696c0cd1111075}{
\index{helpers@{helpers}!pmath\_\-emit@{pmath\_\-emit}}
\index{pmath\_\-emit@{pmath\_\-emit}!helpers@{helpers}}
\paragraph[{pmath\_\-emit}]{\setlength{\rightskip}{0pt plus 5cm}PMATH\_\-API void pmath\_\-emit ({\bf pmath\_\-t} {\em object}, \/  {\bf pmath\_\-t} {\em tag})\hspace{0.3cm}{\tt  \mbox{[}related, inherited\mbox{]}}}\hfill}
\label{group__helpers_ga06135012f4d2a0faf696c0cd1111075}


Emit an object to be gathered by the appropriate surounding \hyperlink{group__helpers_g077f3730ca4275b87d9a35bce6013e45}{pmath\_\-gather\_\-begin()} ... \hyperlink{group__helpers_ga2f732b35703986263e3a15592b4a46e}{pmath\_\-gather\_\-end()} function pair. 

\begin{Desc}
\item[Parameters:]
\begin{description}
\item[{\em object}]The objcet to be emitted, it will be freed. \item[{\em tag}]\hyperlink{class_a}{A} tag object. The sourounding Gather() with a pattern, that matches {\tt tag} will collect the {\tt object}. {\tt tag} will be freed.\end{description}
\end{Desc}
\begin{Desc}
\item[See also:]\hyperlink{group__helpers_g077f3730ca4275b87d9a35bce6013e45}{pmath\_\-gather\_\-begin} 

\hyperlink{group__helpers_ga2f732b35703986263e3a15592b4a46e}{pmath\_\-gather\_\-end} \end{Desc}
\hypertarget{group__helpers_g96660afa0732edf675653eb6ddeab0a4}{
\index{helpers@{helpers}!pmath\_\-build\_\-value\_\-v@{pmath\_\-build\_\-value\_\-v}}
\index{pmath\_\-build\_\-value\_\-v@{pmath\_\-build\_\-value\_\-v}!helpers@{helpers}}
\paragraph[{pmath\_\-build\_\-value\_\-v}]{\setlength{\rightskip}{0pt plus 5cm}PMATH\_\-API {\bf pmath\_\-t} pmath\_\-build\_\-value\_\-v (const char $\ast$ {\em format}, \/  va\_\-list {\em args})\hspace{0.3cm}{\tt  \mbox{[}related, inherited\mbox{]}}}\hfill}
\label{group__helpers_g96660afa0732edf675653eb6ddeab0a4}


Generate a List of objects with a format string. 

\begin{Desc}
\item[Parameters:]
\begin{description}
\item[{\em format}]\hyperlink{class_a}{A} string that specifies the tuple's item's type. \item[{\em args}]\hyperlink{class_a}{A} va\_\-list - variable argument list.\end{description}
\end{Desc}
\begin{Desc}
\item[See also:]\hyperlink{group__helpers_g13a748aa283c5f5408cce037d3ad224d}{pmath\_\-build\_\-value} \end{Desc}
\hypertarget{group__helpers_g13a748aa283c5f5408cce037d3ad224d}{
\index{helpers@{helpers}!pmath\_\-build\_\-value@{pmath\_\-build\_\-value}}
\index{pmath\_\-build\_\-value@{pmath\_\-build\_\-value}!helpers@{helpers}}
\paragraph[{pmath\_\-build\_\-value}]{\setlength{\rightskip}{0pt plus 5cm}PMATH\_\-API {\bf pmath\_\-t} pmath\_\-build\_\-value (const char $\ast$ {\em format}, \/   {\em ...})\hspace{0.3cm}{\tt  \mbox{[}related, inherited\mbox{]}}}\hfill}
\label{group__helpers_g13a748aa283c5f5408cce037d3ad224d}


Generate a List of objects with a format string. 

\begin{Desc}
\item[Parameters:]
\begin{description}
\item[{\em format}]\hyperlink{class_a}{A} string that specifies the tuple's item's type. See below. \item[{\em ...}]The tuple/list items\end{description}
\end{Desc}
The format string characters specify the item's type:\begin{itemize}
\item {\tt b} \mbox{[}int\mbox{]} converts a C int to True or False\end{itemize}


\begin{itemize}
\item {\tt i} \mbox{[}int\mbox{]}\item {\tt l} \mbox{[}long int\mbox{]}\item {\tt k} \mbox{[}long long int\mbox{]}\item {\tt n} \mbox{[}ssize\_\-t\mbox{]}\end{itemize}


\begin{itemize}
\item {\tt I} \mbox{[}unsigned int\mbox{]}\item {\tt L} \mbox{[}unsigned long int\mbox{]}\item {\tt K} \mbox{[}unsigned long long int\mbox{]}\item {\tt N} \mbox{[}size\_\-t\mbox{]}\end{itemize}


\begin{itemize}
\item {\tt f} \mbox{[}double\mbox{]} NaN's and Ininity are converted to Indeterminate and +/-Infinity\end{itemize}


\begin{itemize}
\item {\tt o} \mbox{[}\hyperlink{classpmath__t}{pmath\_\-t}\mbox{]} \hyperlink{class_a}{A} pMath Object, the reference is stolen\end{itemize}


\begin{itemize}
\item {\tt c} \mbox{[}int\mbox{]} convert a C int representing a (unicode) character to a string of length 1.\end{itemize}


\begin{itemize}
\item {\tt s} \mbox{[}char$\ast$\mbox{]} converts a zero-terminated C string to a pMath string using Latin-1 encoding.\end{itemize}


\begin{itemize}
\item {\tt s\#} \mbox{[}char$\ast$,int\mbox{]} takes a char buffer and a length to build a pMath string of that length using Latin-1 encoding.\end{itemize}


\begin{itemize}
\item {\tt z} \mbox{[}char$\ast$\mbox{]} a zero-terminated C string and converts it to a symbol using \hyperlink{group__symbols_g597a971f788584cac3f327f1afdd5f41}{pmath\_\-symbol\_\-find()}.\end{itemize}


\begin{itemize}
\item {\tt u} \mbox{[}char$\ast$\mbox{]} converts a zero-terminated C string to a pMath string using UTF-8 encoding.\end{itemize}


\begin{itemize}
\item {\tt u\#} \mbox{[}char$\ast$,int\mbox{]} takes a char buffer and a length to build a pMath string of that length using UTF-8 encoding.\end{itemize}


\begin{itemize}
\item {\tt U} \mbox{[}uint16\_\-t$\ast$\mbox{]} converts a zero-terminated double-byte C string to a pMath string using UTF-16 encoding. This is generally useful only where sizeof(uint16\_\-t) == sizeof(wchar\_\-t), e.g. on Windows but not on Linux.\end{itemize}


\begin{itemize}
\item {\tt U\#} \mbox{[}uint16\_\-t$\ast$,int\mbox{]} takes a character buffer and a length to build a pMath string of that length using UTF-16 encoding. This is generally useful only on platforms with sizeof(uint16\_\-t) == sizeof(wchar\_\-t), e.g. on Windows but not on Linux.\end{itemize}


\begin{itemize}
\item {\tt C{\em tt\/}} \mbox{[}maching the 2 t's\mbox{]} build a complex value\item {\tt Q{\em tt\/}} \mbox{[}maching the 2 t's\mbox{]} build a rational value from two integers.\end{itemize}


\begin{itemize}
\item {\tt (items)} \mbox{[}matching items\mbox{]} constructs a sublist of items.\end{itemize}


\begin{Desc}
\item[Note:]When the format string denotes only one object, this object will be returned alone. So for a \hyperlink{classpmath__t}{pmath\_\-t} x, pmath\_\-build\_\-value(\char`\"{}o\char`\"{}, x) == x. \par
 If you want to return a list in any case, use \char`\"{}(...)\char`\"{}: \char`\"{}i\char`\"{} gives an integer, \char`\"{}ii\char`\"{} and \char`\"{}(ii)\char`\"{} a list of two integers and \char`\"{}(i)\char`\"{} a list of one integer. \end{Desc}
\hypertarget{group__helpers_g24403dfbd825b17fc4c6da5973922184}{
\index{helpers@{helpers}!pmath\_\-options\_\-extract@{pmath\_\-options\_\-extract}}
\index{pmath\_\-options\_\-extract@{pmath\_\-options\_\-extract}!helpers@{helpers}}
\paragraph[{pmath\_\-options\_\-extract}]{\setlength{\rightskip}{0pt plus 5cm}PMATH\_\-API {\bf pmath\_\-expr\_\-t} pmath\_\-options\_\-extract ({\bf pmath\_\-expr\_\-t} {\em expr}, \/  size\_\-t {\em last\_\-nonoption})\hspace{0.3cm}{\tt  \mbox{[}related, inherited\mbox{]}}}\hfill}
\label{group__helpers_g24403dfbd825b17fc4c6da5973922184}


Extract custom option values from an expression. 

\begin{Desc}
\item[Parameters:]
\begin{description}
\item[{\em expr}]The expression that may custom option values. It wont be freed. \item[{\em last\_\-nonoption}]The index of the last argument that is not an option rule. \end{description}
\end{Desc}
\begin{Desc}
\item[Returns:]\hyperlink{class_a}{A} list of all given option values or NULL on error. You must destroy it.\end{Desc}
Imagine, {\tt expr} = `f(a,b,A-$>$1,B-$>$2)` and {\tt last\_\-nonoption} is 2, then the result value is a list `\{A-$>$1, B-$>$2\}`. You can use this return value as the {\tt extra} parameter in \hyperlink{group__helpers_gc244ab0720278b396976728a39f8bde6}{pmath\_\-option\_\-value()}.

When {\tt last\_\-nonoption} was 1, a message would be generated (b is no rule ...) and the return value is NULL. In that case, the calling function should have no further effects and return. \hypertarget{group__helpers_gc244ab0720278b396976728a39f8bde6}{
\index{helpers@{helpers}!pmath\_\-option\_\-value@{pmath\_\-option\_\-value}}
\index{pmath\_\-option\_\-value@{pmath\_\-option\_\-value}!helpers@{helpers}}
\paragraph[{pmath\_\-option\_\-value}]{\setlength{\rightskip}{0pt plus 5cm}PMATH\_\-API {\bf pmath\_\-t} pmath\_\-option\_\-value ({\bf pmath\_\-t} {\em fn}, \/  {\bf pmath\_\-t} {\em name}, \/  {\bf pmath\_\-t} {\em extra})\hspace{0.3cm}{\tt  \mbox{[}related, inherited\mbox{]}}}\hfill}
\label{group__helpers_gc244ab0720278b396976728a39f8bde6}


Retrieve a option value of a given function. 

\begin{Desc}
\item[Parameters:]
\begin{description}
\item[{\em fn}]The function for which the requested option value is defined. It wont be freed. If it is NULL, the current head (see \hyperlink{group__helpers_g70aa270956b6c8f8eb43431f9775ae88}{pmath\_\-current\_\-head} ) will be used. \item[{\em name}]The name of the option value (in general, a symbol). It wont be freed. \item[{\em extra}]\hyperlink{class_a}{A} list of extra option rules or PMATH\_\-UNDEFINED. It wont be freed. If it is not PMATH\_\-UNDEFINED, it must be a rule (`a-$>$b`, `a:$>$b`) or a list of rules. \end{description}
\end{Desc}
\begin{Desc}
\item[Returns:]The requested option value. \end{Desc}
