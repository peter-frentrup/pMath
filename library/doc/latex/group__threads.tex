\hypertarget{group__threads}{
\subsection{Multithreading with pMath}
\label{group__threads}\index{Multithreading with pMath@{Multithreading with pMath}}
}
The Thread abstraction in pMath.  


\subsubsection*{Data Structures}
\begin{CompactItemize}
\item 
class \hyperlink{classpmath__threadlock__t}{pmath\_\-threadlock\_\-t}
\begin{CompactList}\small\item\em \hyperlink{class_a}{A} reentrant lock for threads. \item\end{CompactList}\item 
class \hyperlink{classpmath__thread__t}{pmath\_\-thread\_\-t}
\begin{CompactList}\small\item\em The Representation of a thread. \item\end{CompactList}\end{CompactItemize}
\subsubsection*{Functions}
\begin{CompactItemize}
\item 
PMATH\_\-API \hyperlink{classpmath__t}{pmath\_\-t} \hyperlink{group__threads_ga545e19cccf64ee4848a5506fa20cf21}{pmath\_\-thread\_\-local\_\-save} (\hyperlink{classpmath__t}{pmath\_\-t} key, \hyperlink{classpmath__t}{pmath\_\-t} value)
\begin{CompactList}\small\item\em Store a thread/thread-local value. \item\end{CompactList}\item 
PMATH\_\-API \hyperlink{classpmath__t}{pmath\_\-t} \hyperlink{group__threads_g64fe008ee1e3a4841eb98ca13e212537}{pmath\_\-thread\_\-local\_\-load} (\hyperlink{classpmath__t}{pmath\_\-t} key)
\begin{CompactList}\small\item\em Load a thread/thread-local value. \item\end{CompactList}\item 
PMATH\_\-API void \hyperlink{group__threads_gf1aa6d6603faaa4120207be6108e356c}{pmath\_\-throw} (\hyperlink{classpmath__t}{pmath\_\-t} exception)
\begin{CompactList}\small\item\em Throw an exception. \item\end{CompactList}\item 
PMATH\_\-API \hyperlink{classpmath__t}{pmath\_\-t} \hyperlink{group__threads_gf791e0088342297d6511a4481421f446}{pmath\_\-catch} (void)
\begin{CompactList}\small\item\em Catch any exception. \item\end{CompactList}\item 
PMATH\_\-API \hyperlink{group__general__types_gc92090cb0b56345d6c379ed2341d4ef4}{pmath\_\-bool\_\-t} \hyperlink{group__threads_gb75a9c87401fddb42b297ddb0495415f}{pmath\_\-aborting} (void)
\begin{CompactList}\small\item\em Queries whether pMath was requested to abort the evaluation of the current thread. \item\end{CompactList}\item 
PMATH\_\-API void \hyperlink{group__threads_g84e45036b76764def6390af12d2070bf}{pmath\_\-abort\_\-please} (void)
\begin{CompactList}\small\item\em Requests pMath to abort the current evaluation. \item\end{CompactList}\item 
\hypertarget{group__threads_g688141bacfb504b83dc319695f243b95}{
PMATH\_\-API void \hyperlink{group__threads_g688141bacfb504b83dc319695f243b95}{pmath\_\-suspend\_\-all\_\-please} (void)}
\label{group__threads_g688141bacfb504b83dc319695f243b95}

\begin{CompactList}\small\item\em Suspend all other threads. This function does not realy suspend threads immediately. Any other thread, that calls \hyperlink{group__threads_gb75a9c87401fddb42b297ddb0495415f}{pmath\_\-aborting()} (or \hyperlink{group__threads_g3932a5c364197c999dfab0aeb28e8779}{pmath\_\-thread\_\-aborting()}), will block until we call \hyperlink{group__threads_gccf30e13ea6e65136fca6294267f00c2}{pmath\_\-resume\_\-all()}. \item\end{CompactList}\item 
PMATH\_\-API void \hyperlink{group__threads_gccf30e13ea6e65136fca6294267f00c2}{pmath\_\-resume\_\-all} (void)
\begin{CompactList}\small\item\em Resume all other threads. \item\end{CompactList}\item 
PMATH\_\-API void \hyperlink{group__threads_gfcbd1d376791cb95a78e102366b7b79a}{pmath\_\-threadlock\_\-t::pmath\_\-thread\_\-call\_\-locked} (\hyperlink{classpmath__threadlock__t}{pmath\_\-threadlock\_\-t} $\ast$threadlock\_\-ptr, \hyperlink{group__general__types_ge1a454657f18f3cc54508adeccccbcbc}{pmath\_\-callback\_\-t} callback, void $\ast$data)
\begin{CompactList}\small\item\em Execute a function synchronized with a threadlock. \item\end{CompactList}\item 
PMATH\_\-API \hyperlink{classpmath__thread__t}{pmath\_\-thread\_\-t} \hyperlink{group__threads_g908b6ee94115539a20530cfb051dc52c}{pmath\_\-thread\_\-t::pmath\_\-thread\_\-get\_\-current} (void)
\begin{CompactList}\small\item\em Get the current pMath thread. \item\end{CompactList}\item 
PMATH\_\-API \hyperlink{classpmath__thread__t}{pmath\_\-thread\_\-t} \hyperlink{group__threads_gb0e137fb921f9b329a8a9e7c4efa3b3d}{pmath\_\-thread\_\-t::pmath\_\-thread\_\-get\_\-parent} (\hyperlink{classpmath__thread__t}{pmath\_\-thread\_\-t} thread)
\begin{CompactList}\small\item\em Get a thread's direct parent. \item\end{CompactList}\item 
PMATH\_\-API \hyperlink{group__general__types_gc92090cb0b56345d6c379ed2341d4ef4}{pmath\_\-bool\_\-t} \hyperlink{group__threads_g4409c7f042a1809eee72329002635fb9}{pmath\_\-thread\_\-t::pmath\_\-thread\_\-is\_\-parent} (\hyperlink{classpmath__thread__t}{pmath\_\-thread\_\-t} parent, \hyperlink{classpmath__thread__t}{pmath\_\-thread\_\-t} child)
\begin{CompactList}\small\item\em Queries whether a thread is one of the parents of another. \item\end{CompactList}\item 
PMATH\_\-API \hyperlink{group__general__types_gc92090cb0b56345d6c379ed2341d4ef4}{pmath\_\-bool\_\-t} \hyperlink{group__threads_g3932a5c364197c999dfab0aeb28e8779}{pmath\_\-thread\_\-t::pmath\_\-thread\_\-aborting} (\hyperlink{classpmath__thread__t}{pmath\_\-thread\_\-t} thread)
\begin{CompactList}\small\item\em Queries whether pMath was requested to abort the evaluation of a specific thread or its parents. \item\end{CompactList}\end{CompactItemize}


\subsubsection{Detailed Description}
The Thread abstraction in pMath. 

pMath stores several data local to a thread. Therefor, it maintains a \hyperlink{classpmath__thread__t}{pmath\_\-thread\_\-t} in every operating system thread it runs on. Those \hyperlink{classpmath__thread__t}{pmath\_\-thread\_\-t} variables are created and freed via \hyperlink{group__frontend_gfb9f2c789bee5295c6794d16c0164943}{pmath\_\-init()} and \hyperlink{group__frontend_g012705e1fd248a7cebf738bae6375dd9}{pmath\_\-done()} respectively. Thus, you have to call those two functions once in every thread that uses pMath functions (and abort the thread if \hyperlink{group__frontend_gfb9f2c789bee5295c6794d16c0164943}{pmath\_\-init()} fails).

pMath Threads can have parents. While one thread is running, its parent thread waits (for all its children) and is effectively immutable. This way, child threads can read their parent thread's local variables.\hypertarget{group__threads_section_thread_syncronization}{}\subsubsection{Synchronization}\label{group__threads_section_thread_syncronization}
In other environments, you normaly do synchronization with mutexes and the like. But if we did so, a deadlock could occur when a mutex is allready locked by the parent thread, which in turn is waiting for its children to finish.

The solution is to use pMath threadlocks: You simply synchronize with a \hyperlink{classpmath__symbol__t}{pmath\_\-symbol\_\-t} through \hyperlink{group__symbols_g95b141d9cb33fba80d6a807f304ee3b7}{pmath\_\-symbol\_\-synchronized()} or directly with a \hyperlink{classpmath__threadlock__t}{pmath\_\-threadlock\_\-t} and \hyperlink{group__threads_gfcbd1d376791cb95a78e102366b7b79a}{pmath\_\-thread\_\-call\_\-locked()}. This is reentrant and locks execution to a given thread {\em and\/} its child threads. pMath cares about avoiding deadlocks behind the scenes.

Note that threadlocks are needed only if the syncronized code might create child threads or calls other code that utilizes thread locks. Threads might be created by \hyperlink{classpmath__t_d95c86ef0de178de4d3560518c8a8157}{pmath\_\-evaluate()} \& co.

In other situations, you should use mutexes/semaphores from your operating system library or spinlocks (see \hyperlink{group__atomic__ops_gf143a22332da6a2065bac14069ecbf7f}{pmath\_\-atomic\_\-lock()} and \hyperlink{group__atomic__ops_ga61bbbab4adc550cc66d1d5f9cf22afd}{pmath\_\-atomic\_\-unlock()} ), because they are much faster.

For some simple changes on global integer/pointer variables, you can use \hyperlink{group__atomic__ops}{Atomic Operations}. 

\subsubsection{Function Documentation}
\hypertarget{group__threads_ga545e19cccf64ee4848a5506fa20cf21}{
\index{threads@{threads}!pmath\_\-thread\_\-local\_\-save@{pmath\_\-thread\_\-local\_\-save}}
\index{pmath\_\-thread\_\-local\_\-save@{pmath\_\-thread\_\-local\_\-save}!threads@{threads}}
\paragraph[{pmath\_\-thread\_\-local\_\-save}]{\setlength{\rightskip}{0pt plus 5cm}PMATH\_\-API {\bf pmath\_\-t} pmath\_\-thread\_\-local\_\-save ({\bf pmath\_\-t} {\em key}, \/  {\bf pmath\_\-t} {\em value})}\hfill}
\label{group__threads_ga545e19cccf64ee4848a5506fa20cf21}


Store a thread/thread-local value. 

\begin{Desc}
\item[Parameters:]
\begin{description}
\item[{\em key}]The key that can be used to obtain the value with \_\-pmath\_\-thread\_\-local\_\-load(). It wont be freed. \item[{\em value}]The thread/thread-local value. It will be freed. \end{description}
\end{Desc}
\begin{Desc}
\item[Returns:]\hyperlink{group__objects_ge2646df76dcb0113715322b13a1f36f0}{PMATH\_\-UNDEFINED} or the previous value that was stored with the same key. You must destroy it.\end{Desc}
Note that keys of the form `symboltag` are used to store whether a message should be suppressed (value PMATH\_\-SYMBOL\_\-OFF) or not (value NULL).

All keys that are \hyperlink{group__objects_ge2646df76dcb0113715322b13a1f36f0}{magic numbers}, have special meanings for \hyperlink{group__threads_ga545e19cccf64ee4848a5506fa20cf21}{pmath\_\-thread\_\-local\_\-save()}. You should not use them as the a key.

Keys which are only symbols are used for thread-local symbols (\begin{Desc}
\item[See also:]\hyperlink{group__symbols_g5d508ec0d32d617b6c642de54907ee17}{pmath\_\-symbol\_\-attributes\_\-t}). \end{Desc}
\hypertarget{group__threads_g64fe008ee1e3a4841eb98ca13e212537}{
\index{threads@{threads}!pmath\_\-thread\_\-local\_\-load@{pmath\_\-thread\_\-local\_\-load}}
\index{pmath\_\-thread\_\-local\_\-load@{pmath\_\-thread\_\-local\_\-load}!threads@{threads}}
\paragraph[{pmath\_\-thread\_\-local\_\-load}]{\setlength{\rightskip}{0pt plus 5cm}PMATH\_\-API {\bf pmath\_\-t} pmath\_\-thread\_\-local\_\-load ({\bf pmath\_\-t} {\em key})}\hfill}
\label{group__threads_g64fe008ee1e3a4841eb98ca13e212537}


Load a thread/thread-local value. 

\begin{Desc}
\item[Parameters:]
\begin{description}
\item[{\em key}]\hyperlink{class_a}{A} key that was used to save the value with \_\-pmath\_\-thread\_\-local\_\-save() before. It wont be freed. \end{description}
\end{Desc}
\begin{Desc}
\item[Returns:]PMATH\_\-UNDEFINED or the stored value. You must destroy the it.\end{Desc}
If there is nothing stored for key in the current thread, its parent threads are processed. If none of them stores something under `key` and key is a symbol, The global value is used. \hypertarget{group__threads_gf1aa6d6603faaa4120207be6108e356c}{
\index{threads@{threads}!pmath\_\-throw@{pmath\_\-throw}}
\index{pmath\_\-throw@{pmath\_\-throw}!threads@{threads}}
\paragraph[{pmath\_\-throw}]{\setlength{\rightskip}{0pt plus 5cm}PMATH\_\-API void pmath\_\-throw ({\bf pmath\_\-t} {\em exception})}\hfill}
\label{group__threads_gf1aa6d6603faaa4120207be6108e356c}


Throw an exception. 

\begin{Desc}
\item[Parameters:]
\begin{description}
\item[{\em exception}]The exception to be thrown. It will be freed. You cannot throw the \hyperlink{group__objects_ge2646df76dcb0113715322b13a1f36f0}{magic number PMATH\_\-UNDEFINED}.\end{description}
\end{Desc}
If there is already an uncought exception, this new exception is lost. \hypertarget{group__threads_gf791e0088342297d6511a4481421f446}{
\index{threads@{threads}!pmath\_\-catch@{pmath\_\-catch}}
\index{pmath\_\-catch@{pmath\_\-catch}!threads@{threads}}
\paragraph[{pmath\_\-catch}]{\setlength{\rightskip}{0pt plus 5cm}PMATH\_\-API {\bf pmath\_\-t} pmath\_\-catch (void)}\hfill}
\label{group__threads_gf791e0088342297d6511a4481421f446}


Catch any exception. 

\begin{Desc}
\item[Returns:]exception The exception to be thrown. If there is no exception available, PMATH\_\-UNDEFINED will be returned.\end{Desc}
If you cannot handle the exception, you can re-throw it with \hyperlink{group__threads_gf1aa6d6603faaa4120207be6108e356c}{pmath\_\-throw()}. \hypertarget{group__threads_gb75a9c87401fddb42b297ddb0495415f}{
\index{threads@{threads}!pmath\_\-aborting@{pmath\_\-aborting}}
\index{pmath\_\-aborting@{pmath\_\-aborting}!threads@{threads}}
\paragraph[{pmath\_\-aborting}]{\setlength{\rightskip}{0pt plus 5cm}PMATH\_\-API {\bf pmath\_\-bool\_\-t} pmath\_\-aborting (void)}\hfill}
\label{group__threads_gb75a9c87401fddb42b297ddb0495415f}


Queries whether pMath was requested to abort the evaluation of the current thread. 

\begin{Desc}
\item[Returns:]Whether the user called \hyperlink{group__threads_g84e45036b76764def6390af12d2070bf}{pmath\_\-abort\_\-please()} or an exception was thrown or a time-out is passed. \end{Desc}
\hypertarget{group__threads_g84e45036b76764def6390af12d2070bf}{
\index{threads@{threads}!pmath\_\-abort\_\-please@{pmath\_\-abort\_\-please}}
\index{pmath\_\-abort\_\-please@{pmath\_\-abort\_\-please}!threads@{threads}}
\paragraph[{pmath\_\-abort\_\-please}]{\setlength{\rightskip}{0pt plus 5cm}PMATH\_\-API void pmath\_\-abort\_\-please (void)}\hfill}
\label{group__threads_g84e45036b76764def6390af12d2070bf}


Requests pMath to abort the current evaluation. 

This function is signal-safe.

\begin{Desc}
\item[See also:]\hyperlink{group__frontend_g4934c2dc54f852627f8b291543a21e43}{pmath\_\-continue\_\-after\_\-abort()} \end{Desc}
\hypertarget{group__threads_gccf30e13ea6e65136fca6294267f00c2}{
\index{threads@{threads}!pmath\_\-resume\_\-all@{pmath\_\-resume\_\-all}}
\index{pmath\_\-resume\_\-all@{pmath\_\-resume\_\-all}!threads@{threads}}
\paragraph[{pmath\_\-resume\_\-all}]{\setlength{\rightskip}{0pt plus 5cm}PMATH\_\-API void pmath\_\-resume\_\-all (void)}\hfill}
\label{group__threads_gccf30e13ea6e65136fca6294267f00c2}


Resume all other threads. 

\begin{Desc}
\item[See also:]\hyperlink{group__threads_g688141bacfb504b83dc319695f243b95}{pmath\_\-suspend\_\-all\_\-please} \end{Desc}
\hypertarget{group__threads_gfcbd1d376791cb95a78e102366b7b79a}{
\index{threads@{threads}!pmath\_\-thread\_\-call\_\-locked@{pmath\_\-thread\_\-call\_\-locked}}
\index{pmath\_\-thread\_\-call\_\-locked@{pmath\_\-thread\_\-call\_\-locked}!threads@{threads}}
\paragraph[{pmath\_\-thread\_\-call\_\-locked}]{\setlength{\rightskip}{0pt plus 5cm}PMATH\_\-API void pmath\_\-thread\_\-call\_\-locked ({\bf pmath\_\-threadlock\_\-t} $\ast$ {\em threadlock\_\-ptr}, \/  {\bf pmath\_\-callback\_\-t} {\em callback}, \/  void $\ast$ {\em data})\hspace{0.3cm}{\tt  \mbox{[}related, inherited\mbox{]}}}\hfill}
\label{group__threads_gfcbd1d376791cb95a78e102366b7b79a}


Execute a function synchronized with a threadlock. 

\begin{Desc}
\item[Parameters:]
\begin{description}
\item[{\em threadlock\_\-ptr}]\hyperlink{class_a}{A} pointer to the threadlock. \item[{\em callback}]The function to be executed when the symbol is locked. \item[{\em data}]\hyperlink{class_a}{A} pointer that will be passed to callback.\end{description}
\end{Desc}
All you have to do is initialize the threadlock {\tt threadlock\_\-ptr} points to with NULL before you call this function for the first time: 

\begin{Code}\begin{verbatim}static pmath_threadlock_t lock = NULL;
...
pmath_thread_call_locked(&lock, my_callback, my_data);
\end{verbatim}
\end{Code}



To synchronize with a symbol, use \hyperlink{group__symbols_g95b141d9cb33fba80d6a807f304ee3b7}{pmath\_\-symbol\_\-synchronized()}. \hypertarget{group__threads_g908b6ee94115539a20530cfb051dc52c}{
\index{threads@{threads}!pmath\_\-thread\_\-get\_\-current@{pmath\_\-thread\_\-get\_\-current}}
\index{pmath\_\-thread\_\-get\_\-current@{pmath\_\-thread\_\-get\_\-current}!threads@{threads}}
\paragraph[{pmath\_\-thread\_\-get\_\-current}]{\setlength{\rightskip}{0pt plus 5cm}PMATH\_\-API {\bf pmath\_\-thread\_\-t} pmath\_\-thread\_\-get\_\-current (void)\hspace{0.3cm}{\tt  \mbox{[}related, inherited\mbox{]}}}\hfill}
\label{group__threads_g908b6ee94115539a20530cfb051dc52c}


Get the current pMath thread. 

\begin{Desc}
\item[Returns:]\hyperlink{class_a}{A} \hyperlink{classpmath__thread__t}{pmath\_\-thread\_\-t}. This is NULL, if you did not register the current thread to pMath via \hyperlink{group__frontend_gfb9f2c789bee5295c6794d16c0164943}{pmath\_\-init()}. \end{Desc}
\hypertarget{group__threads_gb0e137fb921f9b329a8a9e7c4efa3b3d}{
\index{threads@{threads}!pmath\_\-thread\_\-get\_\-parent@{pmath\_\-thread\_\-get\_\-parent}}
\index{pmath\_\-thread\_\-get\_\-parent@{pmath\_\-thread\_\-get\_\-parent}!threads@{threads}}
\paragraph[{pmath\_\-thread\_\-get\_\-parent}]{\setlength{\rightskip}{0pt plus 5cm}PMATH\_\-API {\bf pmath\_\-thread\_\-t} pmath\_\-thread\_\-get\_\-parent ({\bf pmath\_\-thread\_\-t} {\em thread})\hspace{0.3cm}{\tt  \mbox{[}related, inherited\mbox{]}}}\hfill}
\label{group__threads_gb0e137fb921f9b329a8a9e7c4efa3b3d}


Get a thread's direct parent. 

\begin{Desc}
\item[Parameters:]
\begin{description}
\item[{\em thread}]\hyperlink{class_a}{A} thread. \end{description}
\end{Desc}
\begin{Desc}
\item[Returns:]The direct parent of thread. Usualy NULL. \end{Desc}
\hypertarget{group__threads_g4409c7f042a1809eee72329002635fb9}{
\index{threads@{threads}!pmath\_\-thread\_\-is\_\-parent@{pmath\_\-thread\_\-is\_\-parent}}
\index{pmath\_\-thread\_\-is\_\-parent@{pmath\_\-thread\_\-is\_\-parent}!threads@{threads}}
\paragraph[{pmath\_\-thread\_\-is\_\-parent}]{\setlength{\rightskip}{0pt plus 5cm}PMATH\_\-API {\bf pmath\_\-bool\_\-t} pmath\_\-thread\_\-is\_\-parent ({\bf pmath\_\-thread\_\-t} {\em parent}, \/  {\bf pmath\_\-thread\_\-t} {\em child})\hspace{0.3cm}{\tt  \mbox{[}related, inherited\mbox{]}}}\hfill}
\label{group__threads_g4409c7f042a1809eee72329002635fb9}


Queries whether a thread is one of the parents of another. 

\begin{Desc}
\item[Parameters:]
\begin{description}
\item[{\em parent}]\hyperlink{class_a}{A} thread. \item[{\em child}]\hyperlink{class_a}{A} thread. \end{description}
\end{Desc}
\begin{Desc}
\item[Returns:]TRUE, if parent is a parent thread of child or if parent == child. FALSE otherwise.\end{Desc}
It is important to know that a parent thread is never executed in parallel with its children. \hypertarget{group__threads_g3932a5c364197c999dfab0aeb28e8779}{
\index{threads@{threads}!pmath\_\-thread\_\-aborting@{pmath\_\-thread\_\-aborting}}
\index{pmath\_\-thread\_\-aborting@{pmath\_\-thread\_\-aborting}!threads@{threads}}
\paragraph[{pmath\_\-thread\_\-aborting}]{\setlength{\rightskip}{0pt plus 5cm}PMATH\_\-API {\bf pmath\_\-bool\_\-t} pmath\_\-thread\_\-aborting ({\bf pmath\_\-thread\_\-t} {\em thread})\hspace{0.3cm}{\tt  \mbox{[}related, inherited\mbox{]}}}\hfill}
\label{group__threads_g3932a5c364197c999dfab0aeb28e8779}


Queries whether pMath was requested to abort the evaluation of a specific thread or its parents. 

\begin{Desc}
\item[Parameters:]
\begin{description}
\item[{\em thread}]\hyperlink{class_a}{A} thread that should be tested. \end{description}
\end{Desc}
\begin{Desc}
\item[Returns:]Whether the given thread should abort evaluation.\end{Desc}
\begin{Desc}
\item[See also:]\hyperlink{group__threads_gb75a9c87401fddb42b297ddb0495415f}{pmath\_\-aborting} \end{Desc}
