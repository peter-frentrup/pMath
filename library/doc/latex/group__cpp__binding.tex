\hypertarget{group__cpp__binding}{
\subsection{C++ Binding}
\label{group__cpp__binding}\index{C++ Binding@{C++ Binding}}
}
\subsubsection*{Data Structures}
\begin{CompactItemize}
\item 
class \hyperlink{classpmath_1_1_expr}{pmath::Expr}
\begin{CompactList}\small\item\em A wrapper for \hyperlink{classpmath__t}{pmath\_\-t} and drived types.

This class wraps a single \hyperlink{classpmath__t}{pmath\_\-t}, so its size is sizeof(void$\ast$). \item\end{CompactList}\item 
class \hyperlink{classpmath_1_1_string}{pmath::String}
\begin{CompactList}\small\item\em A wrapper for \hyperlink{classpmath__string__t}{pmath\_\-string\_\-t}.

This class provides some string utility functions in addition to \hyperlink{classpmath_1_1_expr}{Expr}. \item\end{CompactList}\item 
class \hyperlink{classpmath_1_1_gather}{pmath::Gather}
\begin{CompactList}\small\item\em Utility class for emitting and gathering expressions/building lists

Gathering begins with the construction of the object and ends with a call to \hyperlink{classpmath_1_1_gather_d686bb5cc8ffc544bff3f7aaa3723061}{end()} or the object destruction. \item\end{CompactList}\end{CompactItemize}


\subsubsection{Detailed Description}
There exists a thin layer to easily use pMath with C++. This is usably prefreable over the C API because it handles reference counting/type checking automatically and leads to less \char`\"{}boilerplate code\char`\"{}

To use it, simply {\tt \#include $<$pmath-cpp.h$>$}. The classes are in the {\tt pmath} namespace.

This namespace also containts numerous helper functions to easily construct expression trees.

\begin{Desc}
\item[\hyperlink{todo__todo000001}{Todo}]Document the Expr helper functions.\end{Desc}
